\RequirePackage{slides-shared}
\title{JGroups: Multicast Messaging}

\begin{document}

\makepreamble


\section{Technology Overview}


% Selected Version History
% - 5.0.0
%     - Multiple message types for different content marshalling
%     - Lightweight thread support on current JDK versions


\begin{frame}{Technology Overview}
    \begin{block}{Goals}
        Provide reliable group messaging mechanism.
    \end{block}

    \bigskip

    \begin{block}{Features}
        \begin{itemize}
            \item Basic group messaging interface.
            \item Groups identified by names.
            \item Messages are byte arrays.
            \item Configurable protocol stack.
            \begin{itemize}
                \item Multiple underlying transports.
                \item Multiple reliability mechanisms.
                \item Multiple membership discovery mechanisms.
                \item Multiple error recovery mechanisms.
                \item ...
            \end{itemize}
        \end{itemize}
    \end{block}

    \bigskip

    \hfill ... \url{http://www.jgroups.org}
\end{frame}


\section{Assignment Part I}


\begin{frame}[fragile]{Assignment}
    \begin{block}{Peer}
        Implement a process that will update a shared hash map.
        \begin{itemize}
            \item The shared hash map is available through \lstinline{SharedHashMap} channel.
            \item The updates are transmitted through \lstinline{UpdateEvent} class.
        \end{itemize}
    \end{block}

\begin{lstlisting}[language=java,style=mini]
import java.io.Serializable;

public class UpdateEvent implements Serializable {
    private static final long serialVersionUID = 0xBAADBAADBAADL;

    public int key;
    public String value;
}
\end{lstlisting}
\end{frame}


\begin{frame}[fragile]{Examples To Begin With ...}
\begin{lstlisting}[language=bash,style=mini]
> git clone http://github.com/d-iii-s/teaching-introduction-middleware.git
\end{lstlisting}
    \begin{block}{Java}
\begin{lstlisting}[language=bash,style=mini]
> cd teaching-introduction-middleware/src/jgroups-basic-peer/java
> cat README.md
\end{lstlisting}
    \end{block}
\end{frame}


\section{Interface Overview}


\begin{frame}[fragile]{JChannel Class}
\begin{lstlisting}[language=java,style=mini]
public class JChannel implements Closeable {
    public JChannel ();
    public JChannel (String properties);
    public JChannel (InputStream configuration);

    public JChannel connect (String cluster_name);
    public JChannel disconnect ();

    public JChannel send (Message msg);
    public JChannel send (Address dst, Object obj);
    public JChannel send (Address dst, byte [] buf);

    public JChannel setReceiver (Receiver r);
    public Receiver getReceiver ();

    public View getView ();

    public JChannel addChannelListener (ChannelListener listener);
    public JChannel removeChannelListener (ChannelListener listener);

    ...
}
\end{lstlisting}
\end{frame}


\begin{frame}[fragile]{Message Interface}
\begin{lstlisting}[language=java,style=mini]
public interface Message {
    public Address getDest ();
    public Message setDest (Address new_dest);
    public Address getSrc ();
    public Message setSrc (Address new_src);

    ...
}
\end{lstlisting}
\end{frame}


\begin{frame}[fragile]{BytesMessage Class}
\begin{lstlisting}[language=java,style=mini]
public class BytesMessage implements Message {
    public BytesMessage ();
    public BytesMessage (Address dest);
    public BytesMessage (Address dest, byte [] array);
    public BytesMessage (Address dest, byte [] array, int offset, int length);

    public int getOffset ();
    public int getLength ();
    public byte [] getArray ();
    public BytesMessage setArray (byte [] b, int offset, int length);

    ...
}
\end{lstlisting}
\end{frame}


\begin{frame}[fragile]{ObjectMessage Class}
\begin{lstlisting}[language=java,style=mini]
public class ObjectMessage implements Message {
    public ObjectMessage ();
    public ObjectMessage (Address dest);
    public ObjectMessage (Address dest, Object obj);

    public <T extends Object> T getObject ();
    public ObjectMessage setObject (Object obj);

    ...
}
\end{lstlisting}
\end{frame}


\begin{frame}[fragile]{Receiver Interface}
\begin{lstlisting}[language=java,style=mini]
public interface Receiver {
    default void receive (Message msg);
    default void receive (MessageBatch batch);

    default void viewAccepted (View new_view);

    default void block ();
    default void unblock ();

    default void setState (InputStream input);
    default void getState (OutputStream output);
}
\end{lstlisting}
\end{frame}


\begin{frame}[fragile]{ChannelListener Interface}
\begin{lstlisting}[language=java,style=mini]
public interface ChannelListener {
    public void channelClosed (JChannel channel);
    public void channelConnected (JChannel channel);
    public void channelDisconnected (JChannel channel);
}
\end{lstlisting}
\end{frame}


\begin{frame}{Code Now ...}
    \commitstrip{http://www.commitstrip.com/en/2018/11/20/one-final-detail}{final-detail}{0.6}
\end{frame}


\section{Assignment Part II}


\begin{frame}[fragile]{Assignment}
    \begin{block}{Peer}
        Implement a process that will track and display a shared hash map state.
        \begin{itemize}
            \item The shared hash map is available through \lstinline{SharedHashMap} channel.
            \item The updates are transmitted through \lstinline{UpdateEvent} class.
        \end{itemize}
    \end{block}

\begin{lstlisting}[language=java,style=mini]
import java.io.Serializable;

public class UpdateEvent implements Serializable {
    private static final long serialVersionUID = 0xBAADBAADBAADL;

    public int key;
    public String value;
}
\end{lstlisting}

    \begin{block}{Quiz}
        \begin{itemize}
            \item How would you go about measuring the cluster throughput~?
            \item Will the entire cluster see the same state~?
        \end{itemize}
    \end{block}
\end{frame}


\begin{frame}{Submission}
    \begin{block}{GitLab}
        Use your personal GitLab repository under \\
        \url{https://gitlab.mff.cuni.cz/teaching/nswi163/2021}.
    \end{block}
    \begin{block}{Requirements}
        \begin{itemize}
            \item Use the assignment subdirectory.
            \item Include build scripts and README with instructions.
            \item Do not commit binaries or temporary build artifacts.
        \end{itemize}
    \end{block}
\end{frame}


\end{document}
