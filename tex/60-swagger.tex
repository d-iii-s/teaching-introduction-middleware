\RequirePackage{slides-shared}
\title{Swagger: REST API Generation}

\begin{document}

\makepreamble


\section{Technology Overview}


% TODO Include information on links especially in light of the assignment.
% - A bit on philosophy of using links rather than simply keys (but here links are simply inserted verbatim)
%   https://cloud.google.com/blog/products/application-development/api-design-why-you-should-use-links-not-keys-to-represent-relationships-in-apis
% - Support for linking information in the specification (here keys are still transported but specification says how to make links from them)
%   https://swagger.io/docs/specification/links


\begin{frame}{REST: Representational State Transfer}
    \begin{block}{Features}
        REST compliant web services allow requesting systems to access
        and manipulate textual representations of web resources using
        a uniform and predefined set of stateless operations.

        \hfill ... Wikipedia
    \end{block}

    \bigskip

    Practically: each object (for example each database record) has its own URL
    and each action on the object a specific method or a specific child URL.

    \medskip

    \begin{itemize}
        \item List people with GET at \url{http://example.com/people}
        \item Add new person with POST at \url{http://example.com/people}
        \item Get person info with GET at \url{http://example.com/people/42}
        \item Update person info with POST at \url{http://example.com/people/42}
        \item Delete person info with DELETE at \url{http://example.com/people/42}
    \end{itemize}

\end{frame}


\begin{frame}{REST: Motivation}
    \begin{block}{Motivation}
        Strike balance between \\ need for \emph{explicit interfaces} \\ and need for \emph{loose coupling}.
    \end{block}

    \medskip

    \begin{itemize}
        \item Standard communication protocol (HTTP)
        \begin{itemize}
            \item Already defines CRUD operations
            \item Provides security and reliability
            \item Is easy to deploy across internet
        \end{itemize}
        \item Encourages separating model from view
        \item Supports independent implementation technology \\ between client and server
    \end{itemize}
\end{frame}


\begin{frame}{REST and CRUD}
    \begin{block}{CRUD}
        \begin{description}
             \item[Create] to create an object
             \item[Read] to query object attributes
             \item[Update] to update object attributes
             \item[Delete] to delete an object
        \end{description}
    \end{block}

    \medskip

    \begin{itemize}
        \item The recommended minimum set of operations
        \item Corresponds reasonably well to HTTP methods
        \item Anything beyond CRUD is not considered pure REST
    \end{itemize}
\end{frame}


\begin{frame}[fragile]{REST: Data Transfer}

    Data exchange format is application specific but there are obvious choices
    \begin{itemize}
        \item JSON because of JavaScript in the browser
        \item XML because of existing library support
    \end{itemize}

\begin{lstlisting}[language=json,style=mini]
{
    "name": "Jane Doe",
    "email":  "jane.doe@example.com",
    "url": [
        "http://example.com/~jane.doe",
        "http://example.com/people/jane.doe"
    ],
    "address": {
        "street1": "Our Street One",
        "street2": "Street Line Two",
        "city": "The City",
        "postal": "12345"
    },
    "room": 123
}
\end{lstlisting}
\end{frame}


\begin{frame}{Swagger: API Development for REST}
    \begin{block}{Interface Description}
        \begin{description}
            \item[Paths] to identify data model classes
            \item[Actions] to operate on class instances
            \item[Attributes] with types to describe class instances
            \item[Security] defines access rules
            \item[Comments] provide human readable description
        \end{description}
    \end{block}

    \medskip

    \begin{itemize}
        \item Code generation
        \begin{itemize}
            \item Client libraries
            \item Server stubs
            \item Documentation
            \item Miscellaneous
        \end{itemize}
        \item Editor at \url{http://editor.swagger.io}
        \item Development environment at \url{http://app.swaggerhub.com}
    \end{itemize}
\end{frame}


\section{Assignment Details}


\begin{frame}{Assignment}
    \begin{block}{Inventory Application}
        Keeps track of \emph{users} and \emph{assets}. \\
        Basic user related operations are already defined. \\
        Define similar operations for assets and implement everything.
    \end{block}

    \begin{itemize}
        \item Interface
        \begin{itemize}
            \item Elementary CRUD operations for assets
            \item One to many relationship between users and assets
        \end{itemize}
        \item Server
        \begin{itemize}
            \item Python implementation using Flask, or
            \item Java implementation using Spring
        \end{itemize}
        \item Client
        \begin{itemize}
            \item TypeScript implementation using Angular, or
            \item R and bash helper scripts
        \end{itemize}
    \end{itemize}
\end{frame}


\begin{frame}[fragile]{Assignment Interface: Prologue}
\begin{lstlisting}[language=yaml,style=mini]
swagger: "2.0"

info:
  description: Inventory database
  version: 1.0.0
  title: Inventory
  termsOfService: ""
  license:
    name: Apache 2.0
    url: "http://www.apache.org/licenses/LICENSE-2.0.html"

host: localhost:8080 /@ \hfill \# Simplifies usage of generated code @/

basePath: /v1 /@ \hfill \# Version the API from the start @/

schemes:
  - http /@ \hfill \# Production should use TLS @/

basePath: /v1
\end{lstlisting}
\end{frame}


\begin{frame}[fragile]{Assignment Interface: Listing Users}
\begin{lstlisting}[language=yaml,style=mini]
paths:
  /users:
    get:
      summary: List all users
      operationId: readUsers /@ \hfill \# Operation name in code @/
      produces: [ "application/json" ]
      responses:
        200:
          description: Success
          schema:
            type: array
            items:
              $ref: "#/definitions/UserBase"

definitions:
  UserBase:
    type: object
    properties:
      id: { type: integer }
      firstname: { type: string, description: First name }
      lastname: { type: string, description: Last name }
\end{lstlisting}
\end{frame}


\begin{frame}[fragile]{Assignment Interface: Querying User Data}
\begin{lstlisting}[language=yaml,style=mini]
paths:
  /users/{user_id}:
    get:
      summary: Query user
      operationId: readUser
      parameters:
        - in: path
          name: user_id
          description: User identifier
          required: true
          type: integer
      produces:
        - "application/json"
      responses:
        200:
          description: Success
          schema:
            $ref: "#/definitions/User"
\end{lstlisting}
\end{frame}


\begin{frame}[fragile]{Assignment Interface: Updating User Data}
\begin{lstlisting}[language=yaml,style=mini]
paths:
  /users/{user_id}:
    post:
      summary: Update user
      operationId: updateUser
      consumes: [ "application/json" ]
      produces: [ "application/json" ]
      parameters:
        - in: path
          name: user_id
          description: User identifier
          required: true
          type: integer
        - in: body
          name: body
          description: User data
          required: true
          schema:
            $ref: "#/definitions/User"
      responses:
        405:
          description: Invalid input
\end{lstlisting}
\end{frame}


\begin{frame}[fragile]{Assignment Interface: Inheritance}
\begin{lstlisting}[language=yaml,style=mini]
definitions:
  UserBase:
    type: object
    properties:
      id: { type: integer }
      firstname: { type: string, description: First name }
      lastname: { type: string, description: Last name }
  User:
    allOf:
      - $ref: "#/definitions/UserBase"
      - type: object
        properties:
          mail: { type: string, description: Mail }
          homepage: { type: string, description: Homepage }
          department: { type: string, description: Department }
          phone:
            type: array
            items:
              type: string
            description: Phone numbers
\end{lstlisting}
\end{frame}


\begin{frame}{Code Now ...}
    \commitstrip{http://www.commitstrip.com/en/2016/08/25/a-very-comprehensive-and-precise-spec}{spec-is-code}{0.6}
\end{frame}


\begin{frame}{Assignment Details}
    \begin{block}{Interface}
        Extend with operations and definitions related to assets.
        \begin{itemize}
            \item Same operations as already exist for users
            \item Additionally querying assets per user
        \end{itemize}
    \end{block}

    \medskip

    \begin{block}{Server}
        Pick one and extend it with asset related operations.
    \end{block}

    \medskip

    \begin{block}{Client}
        Pick one and extend it as suggested.
        \begin{itemize}
            \item Angular: All asset operations and per user listing
            \item bash: Population and per user asset listing
            \item R: Plot average asset cost per department
        \end{itemize}
    \end{block}
\end{frame}


\begin{frame}{Submission}
    \begin{block}{GitLab}
        Use your personal GitLab repository under \\
        \url{https://gitlab.mff.cuni.cz/teaching/nswi163/2020}.
    \end{block}
    \begin{block}{Requirements}
        \begin{itemize}
            \item Use assignment subdirectory.
            \item Include build scripts and README with instructions.
            \item Do not commit binaries or temporary build artifacts.
        \end{itemize}
    \end{block}
\end{frame}

\end{document}
