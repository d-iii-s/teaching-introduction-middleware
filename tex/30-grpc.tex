\RequirePackage{slides-shared}
\title{gRPC: Remote Procedure Call}

\begin{document}

\makepreamble

\section{Technology Overview}


% https://github.com/grpc/grpc/releases
%
% - 1.50
% - 1.28
%     - Support for client side load balancing across server clusters
% - 1.17
%     - Support for client and server side interceptors


\begin{frame}{Technology Overview}
    \begin{block}{Goals}
        Provide platform independent remote procedure call mechanism.
    \end{block}

    \bigskip

    \begin{block}{Features}
        \begin{itemize}
            \item Protocol buffers as interface description language.
            \item Stub code generation for multiple languages \\
                  (C++, Java, Python, Go, Ruby, JavaScript, PHP, C\# ...).
            \item Binary transport format with compact data representation.
            \item Supports streaming arguments during remote call.
            \item Synchronous and asynchronous invocation code.
            \item Compression support at transport level.
            \item Security support at transport level.
        \end{itemize}
    \end{block}

    \bigskip

    \hfill ... \url{http://www.grpc.io}
\end{frame}


\begin{frame}[fragile]{Examples To Begin With ...}
\begin{lstlisting}[language=bash,style=mini]
> git clone http://github.com/d-iii-s/teaching-introduction-middleware.git
\end{lstlisting}
    \begin{block}{C}
\begin{lstlisting}[language=bash,style=mini]
> cd teaching-introduction-middleware/src/grpc-basic-server/c
> cat README.md
\end{lstlisting}
    \end{block}
    \begin{block}{Java}
\begin{lstlisting}[language=bash,style=mini]
> cd teaching-introduction-middleware/src/grpc-basic-server/java
> cat README.md
\end{lstlisting}
    \end{block}
    \begin{block}{Python}
\begin{lstlisting}[language=bash,style=mini]
> cd teaching-introduction-middleware/src/grpc-basic-server/python
> cat README.md
\end{lstlisting}
    \end{block}
\end{frame}


\begin{frame}[fragile]{Service Specification Example}
\begin{lstlisting}[style=mini]
syntax = "proto3";

message AnExampleRequest { ... }
message AnExampleResponse { ... }

service AnExampleService {

    rpc OneToOneCall (AnExampleRequest) returns (AnExampleResponse) { }

    rpc OneToStreamCall (AnExampleRequest)
        returns (stream AnExampleResponse) { }

    rpc StreamToStreamCall (stream AnExampleRequest)
        returns (stream AnExampleResponse) { }
}
\end{lstlisting}
\end{frame}


\section{Assignment Part I}


\begin{frame}{Assignment}
    \begin{block}{Server}
        Implement a server that will provide information on current time.
        \begin{itemize}
            \item The server should accept a spec of what fields to return.
            \item Fields should be standard YYYY-MM-DD HH:MM:SS.
        \end{itemize}
    \end{block}

    \begin{block}{Client}
        Implement a client that will query server time:
        \begin{itemize}
            \item Pick a random combination of fields.
            \item Query information on current time.
            \item Print the time.
        \end{itemize}
    \end{block}

    \begin{block}{Interoperability}
        Implement compatible clients and servers in two languages.
    \end{block}
\end{frame}


\section{Server Implementation}


\begin{frame}[fragile]{C++ Service Basics}
    \begin{block}{Implementation}
\begin{lstlisting}[language=c,style=mini]
class MyService : public AnExampleService::Service {
    grpc.Status OneToOne (grpc.ServerContext *context,
        const AnExampleRequest *request, AnExampleResponse *response) {
        // Method implementation goes here ...
        return (grpc.Status::OK);
    }
    ...
\end{lstlisting}
    \end{block}
    \begin{block}{Execution}
\begin{lstlisting}[language=c,style=mini]
MyService service;
grpc.ServerBuilder builder;
builder.AddListeningPort ("localhost:8888", grpc.InsecureServerCredentials ());
builder.RegisterService (&service);
std::unique_ptr<grpc.Server> server (builder.BuildAndStart ());

server->Wait ();
\end{lstlisting}
    \end{block}
\end{frame}


\begin{frame}[fragile]{Java Service Basics}
    \begin{block}{Implementation}
\begin{lstlisting}[language=java,style=mini]
class MyService extends AnExampleServiceGrpc.AnExampleServiceImplBase {
    @Override public void OneToOne (
        AnExampleRequest request,
        io.grpc.stub.StreamObserver<AnExampleResponse> responseObserver) {
        // Method implementation goes here ...
        responseObserver.onNext (response);
        responseObserver.onCompleted ();
    }
    ...
\end{lstlisting}
    \end{block}
    \begin{block}{Execution}
\begin{lstlisting}[language=java,style=mini]
io.grpc.Server server = io.grpc.ServerBuilder
    .forPort (8888).addService (new MyService ()).build ().start ();

server.awaitTermination ();
\end{lstlisting}
    \end{block}
\end{frame}


\begin{frame}[fragile]{Python Service Basics}
    \begin{block}{Implementation}
\begin{lstlisting}[language=python,style=mini]
class MyServicer (AnExampleServiceServicer):
    def OneToOne (self, request, context):
        # Method implementation goes here ...
        return response
\end{lstlisting}
    \end{block}

    \bigskip

    \begin{block}{Execution}
\begin{lstlisting}[language=python,style=mini]
server = grpc.server (
    futures.ThreadPoolExecutor (
        max_workers = SERVER_THREAD_COUNT))
add_AnExampleServiceServicer_to_server (MyServicer (), server)
server.add_insecure_port ("localhost:8888")
server.start ()
\end{lstlisting}
    \end{block}
\end{frame}


\section{Client Implementation}


\begin{frame}[fragile]{C++ Client Basics}
    \begin{block}{Connection}
\begin{lstlisting}[language=c,style=mini]
std::shared_ptr<grpc.Channel> channel = grpc.CreateChannel (
    "localhost:8888", grpc.InsecureChannelCredentials ());
\end{lstlisting}
    \end{block}

    \bigskip

    \begin{block}{Invocation}
\begin{lstlisting}[language=c,style=mini]
grpc.ClientContext context;
AnExampleResponse response;
std::shared_ptr<AnExampleService::Stub> stub = AnExampleService::NewStub (channel);
grpc.Status status = stub->OneToOne (&context, request, &response);
if (status.ok ()) {
    // Response available here ...
}
\end{lstlisting}
    \end{block}
\end{frame}


\begin{frame}[fragile]{Java Client Basics}
    \begin{block}{Connection}
\begin{lstlisting}[language=java,style=mini]
io.grpc.ManagedChannel channel = io.grpc.ManagedChannelBuilder
    .forAddress ("localhost", 8888)
    .usePlaintext ()
    .build ();
\end{lstlisting}
    \end{block}

    \bigskip

    \begin{block}{Invocation}
\begin{lstlisting}[language=java,style=mini]
AnExampleServiceGrpc.AnExampleServiceBlockingStub stub =
    AnExampleServiceGrpc.newBlockingStub (channel);
AnExampleResponse response = stub.oneToOne (request);
// Response available here ...
\end{lstlisting}
    \end{block}
\end{frame}


\begin{frame}[fragile]{Python Client Basics}
    \begin{block}{Connection}
\begin{lstlisting}[language=python,style=mini]
with grpc.insecure_channel ("localhost:8888") as channel:
\end{lstlisting}
    \end{block}

    \bigskip

    \begin{block}{Invocation}
\begin{lstlisting}[language=python,style=mini]
stub = AnExampleServiceStub (channel)
response = stub.OneToOne (request)
# Response available here ...
\end{lstlisting}
    \end{block}
\end{frame}


\begin{frame}[fragile]{Show Your Code ...}
    \begin{block}{Query Host Name}
\begin{lstlisting}[language=bash,style=mini]
> hostname
u1-22
\end{lstlisting}
    \end{block}
    \begin{block}{Run Screen Sharing}
\begin{lstlisting}[language=bash,style=mini]
> x11vnc -viewonly
\end{lstlisting}
    \end{block}
\end{frame}


\section{Assignment Part II}


\begin{frame}{Assignment}
    \begin{block}{Performance}
        Measure the performance of your implementation.
    \end{block}

    \bigskip

    \begin{block}{Experiment Design}
        Stick to the following, or provide arguments for why not:
        \begin{itemize}
            \item Random field mix, each field with probability 1/2.
            \item Measure at least two minutes long traffic.
            \item Report average invocation throughput.
            \item No printing during measurement.
            \item Compare with past assignments.
        \end{itemize}
    \end{block}
\end{frame}


\begin{frame}{Submission}
    \begin{block}{GitLab}
        Use your personal GitLab repository under \\
        \url{https://gitlab.mff.cuni.cz/teaching/nswi163/2022}.
    \end{block}
    \begin{block}{Requirements}
        \begin{itemize}
            \item Use the assignment subdirectory.
            \item Write brief report in \lstinline{SOLUTION.md}.
            \item Include build scripts with instructions.
            \item Do not commit binaries or temporary build artifacts.
            \item Tag your solution with \lstinline{task-03} and push the tag.
        \end{itemize}
    \end{block}
\end{frame}


\end{document}
